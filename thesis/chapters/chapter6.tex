\chapter{Conclusion}

	In this work a novel Monte Carlo method, Flat Scan Sampling, to estimate the Joint Density of States of the Ising model was presented. It combines some aspectes from both RPS and WL sampling. A C++ implementation was proposed with two main parameters, REP and skip. FSS can correctly find all of the points in the phase space of an Ising system. The error and mean deviation of the computations vanish when REP is increased and FSS behaves like a standard Monte Carlo method, i.e. the standard deviation converges to zero with the inverse of the square-root of REP. The skip parameter reduces statistical correlation between samples, reducing also the error in the results. When the REP is lower, these effects are more pronounced. Moreover, the wall time of FSS scales linear with REP and skip and a formula to estimate single core wall time is proposed with success.

	A parallel version of the Flat Scan Sampling C++ implementation is also presented and developed with success. With this implementation, studies of parallel scalability were preformed up to 320 computing cores. By fitting Amdahl's Law to experimental results, a value for $p=0.98$ was obtained, meaning that 98\% of the algorithm can be perfectly parallelized. This translates to gains up to 48 times of the performance of single core computations that	 can be achieved.

	A comparison with Wang-Landau sampling was performed to test convergence, accuracy and performance of both methods. The error in Wang-Landau's computations is prominent and never vanishes, due to biased samples at the beginning of the simulation. With FSS, the error does vanish for high REP, costing computing time. As mentioned before, FSS behaves like a standard Monte Carlo method and in this study, it was confirmed that WL does not. It's mean deviation is not linear with the inverse of the square-root of the number of samples nor it converge to 0. Wang-Landau's wall time is linear with the random walk steps like FSS. Finally, FSS will have in average better precision per computing time than WL method. This way, Flat Scan Sampling is a better method to obtain fairly quick and very precise computations while with Wang-Landau we can get an inaccurate approximation very rapidly.

	At the end, the differences between mean and exact thermodynamic variables and some methods to estimate Curie temperatures $T_C$ for the infinite lattice were discussed.Mean thermodynamic variables, i.e., variables computed through ensemble averages, have a smoother curve thus a less defined phase transition region resulting in exact but inaccurate results. On the other hand, exact thermodynamic quantities, the phase transition is more visible meaning that critical temperatures are more precise. Estimating the critical temperature for the infinite lattice can be achieved through the linear fitting of $T_C$ values from the magnetization plots or the Binder cumulant. The Binder cumulant method gave better results since it can estimate, with a high degree of accuracy, the $T_C$ for the infinite lattice with results from smaller lattices. The linear fitting can be improved by having some results from the L32 or L64 lattice.

	This method could easily be extended to the SpinS Ising model where instead of spin-1/2 particles we would consider spin-S particles. In theory we can get better simulate real materials since not all atoms in materials have a total angular momentum of 1/2. 
	
	A GPU/CUDA implementation of Flat Scan Sampling would also be of interest since the method is highly parallelizable and using thousands of GPU computing cores solutions to complicated and larger systems could be achieved with ease.
	
	
	
	